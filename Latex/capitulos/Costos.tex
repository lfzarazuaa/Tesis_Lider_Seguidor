% !TeX encoding = ISO-8859-1
\chapter{An�lisis de Costo}
%\blindtext
\section{Costos variables}
Los costos variables corresponden a los costos directos involucrados en la producci�n y venta de un art�culo. Aline�ndose al presente proyecto, no se cubrieron gastos por ventas, pero si por producci�n, deriv�ndose en las siguientes dos categor�as:

\begin{itemize}
	\item Costos por materias primas.
	\item Costos por mano de obra directa.
\end{itemize}

Los \textit{costos primarios} corresponden a los costos por materia prima directamente gastados. Ya que en este proyecto no se manufactur� ni tampoco se transform� el material, la materia prima corresponde a todo el material utilizado para llevar a cabo el desarrollo del proyecto, abarcando desde la compra de los robots hasta la compra de tornillos, pegamento, etc.

%%Tabla 1
\begin{table}[H]
	\centering
	\begin{tabular}{|p{4.5cm}|c|p{2.5cm}|p{2.5cm}|p{4cm}|}
		\hline
		Articulo & Cantidad & Precio unitario & Precio total \\
		\hline
		TurtleBot3 Burger & 2 & 590 USD* & 1180 USD*\\
		\hline
		Teclado y mouse Logitech & 2 & 343.97 MXN & 687.80 MXN\\
		\hline
		Cargador Energizer & 1 & 206.03 MXN & 206.03 MXN\\
		\hline
		Plan de remplazo & 2 & 68.10 MXN & 136.20 MXN\\
		\hline
		Adaptador de HDMI a VGA & 2 & 188.70 MXN & 188.70 MXN\\
		\hline
		Bater�as & 1 & 76.73 MXN & 73.73 MXN\\
		\hline
		Placas de madera & 3 & 170.00 MXN & 510.00 MXN\\
		\hline
		Tira de pino & 6 & 25.00 MXN & 150.00 MXN\\
		\hline
		Otros (tornillos, pegamento, etc) & n & 150.00 MXN & 150.00 MXN\\
		\hline
		\multicolumn{3}{|c|}{Total} & \cellcolor{naranja}24,808.75 MXN\\
		\hline
	\end{tabular}
	\caption{Tabla de costos variables.}
	\label{tab:Tabla_CostosVariables}
\end{table}
%-------------------------
*La compra de los TurtleBot3 Burger se realiz� el 14 de septiembre de 2018, ese d�a el tipo de cambio correspondi� a 19.24 pesos mexicanos por cada d�lar estadounidense.

\section{Costos fijos}

%\subsection{Costos fijos por horas trabajadas}

%Para calcular los costos fijos por horas trabajadas, se dividen los costos fijos de la empresa entre el total de horas trabajadas.
Los \textit{costos fijos} se presentan a continuaci�n considerando las horas trabajadas.

%%Tabla 2-------------------------------------------------------
\begin{table}[H]
	\centering
	\begin{tabular}{|c|c|c|}
		\hline
		\multirow{2}{*}{TT1} &Total de d�as trabajados & Horas trabajadas por d�a\\ \cline{2-3}
		& 85 & 3\\
		\hline
		\multicolumn{2}{|c|}{Total de horas trabajadas} & 255\\
		\hline
	\end{tabular}
	\caption{D�as trabajados en TT1.}
	\label{tab:Diastrabajados_TT1}
\end{table}
\begin{table}[H]
	\centering
	\begin{tabular}{|c|c|c|}
		\hline
		\multirow{2}{*}{TT2} &Total de d�as trabajados & Horas trabajadas por d�a\\ \cline{2-3}
		& 95 & 4\\
		\hline
		\multicolumn{2}{|c|}{Total de horas trabajadas} & 380 \\
		\hline
	\end{tabular}
	\caption{D�as trabajados TT2.}
	\label{tab:Diastrabajados_TT2}
\end{table}

\begin{table}[H]
	\centering
	\begin{tabular}{|c|c|}
		\hline
		Total de d�as trabajados & 180 d�as \\
		\hline
		Total de horas trabajadas & 635 horas \\
		\hline
	\end{tabular}
	\caption{Total de d�as y horas trabajadas.}
	\label{tab:Total_dias_horas}
\end{table}

%%---------------------------------------------------------------
Nota: El total de d�as trabajados es ideal, ya que �nicamente se est�n contabilizando los 5 primeros d�as de la semana, sin contar fines de semana, ni vacaciones, aunque se haya trabajado en fines de semana y vacaciones. Este an�lisis se hizo bas�ndose en el calendario del Instituto Polit�cnico Nacional (semestres 2019 -1 y 2019 -2).

Los costos fijos son costos peri�dicos, es decir, suelen incurrirse a ellos por medio del tiempo transcurrido, como lo ser�a costos por renta, mantenimiento, etc., pero como en este caso, no se cubrieron ninguno de esos gastos porque fueron gastos cubiertos por el Instituto Polit�cnico Nacional, se consideran como gastos fijos, los gastos de transporte, comidas, y tiempo de estancia en la escuela por todos los miembros del equipo.

%%----------------------------------Tabla 3
\begin{table}[H]
	\centering
	\begin{tabular}{|c|c|c|}
		\hline
		Tipo de gasto & Gasto por d�a [MXN]& Gasto por el total de d�as trabajados (180) \\
		\hline
		Pasajes & \$140.00 & \$25,200.00 \\
		\hline
		Comidas & \$160.00 & \$28,800.00 \\
		\hline
		Otros gastos & \$15.00 & \$2,700.00 \\
		\hline
		\multicolumn{2}{|c|}{TOTAL} & \$56,700.00 MXN \\
		\hline
	\end{tabular}
	\caption{Costo total de gastos fijos.}
	\label{tab:CostoTotal_GastosFijos}
\end{table}

\section{Costo total}

El \textit{costo total} del proyecto corresponde a la suma del \textit{costo fijo} m�s la suma del \textit{costo variable}.
\begin{center}
	\textit{Costo total = Costo fijo + Costo variable}
	
	Costo total del proyecto = \textcolor{red}{\$81,508.75.00}
	
\end{center}
Este es el \textit{precio neto} del proyecto, por lo que si se pensara en comercializarlo, se le podr�a agregar un porcentaje de ganancia con base en el costo total de proyecto. 
% !TeX encoding = ISO-8859-1
\chapter{Glosario}

\textbf{ROS:} Por sus siglas en ingl�s \textit{Robot Operating System} (Sistema Operativo Rob�tico). Es un sistema operativo para robots que pos� herramientas, librer�as y convenciones para ayudar a los desarrolladores de software a crear aplicaciones, con el objetivo de simplificar la tarea de crear complejos y robustos sistemas rob�ticos, proveyendo de varios ambientes de desarrollo especializado en programas y aplicaciones rob�ticas [\citenum{ROS_book}].\\


\textbf{Nodo en ROS:} Un \textit{nodo} es un proceso que ejecuta un c�lculo, el cual se combina en un gr�fico y se comunica con otros nodos mediante \textit{t�picos} en ejecuci�n. Un nodo esta dise�ado para operar en una escala finita. Un ejemplo de como funcionan los nodos se podr�a ver en un sistema para controlar un robot; existir� un nodo para cada acci�n del robot, es decir, habr� un nodo que controle cada sensor que el robot tenga, habr� otro nodo que controle los motores del robot, la localizaci�n, la planificaci�n de la trayectoria, har� un nodo m�s que proporcione una vista gr�fica del sistema y as�, de la misma manera para que cada acci�n del robot.\\


\textbf{T�pico:} Un \textit{t�pico} es el nombre con el cual se identifican el contenido de un \textit{mensaje}, es decir, literalmente un \textit{t�pico} es un tema de conversaci�n [\citenum{ROS_book}]. Un nodo que est� interesado en un determinado tipo de dato se \textit{suscribe} al t�pico correspondiente, en donde puede haber varios publicadores y suscriptores concurrentes en un mismo t�pico. De igual manera un s�lo nodo puede publicar y/o suscribirse a m�ltiples t�picos con el fin de extraer todos los datos que el nodo necesite [\citenum{U_P_Cartagena}].\\


\textbf{Mensaje en ROS:} Los \textit{nodos} se comunican entre s� mediante la publicaci�n de \textit{mensajes} a los t�picos. Un \textit{mensaje} es una estructura de datos simple, que comprende campos escritos [\citenum{Topics}]. En otras palabras, la forma en que un nodo env�a o recibe informaci�n es a trav�s de un \textit{mensaje}. Los mensajes son variables como enteros, puntos  flotantes y booleanos [\citenum{ROS_book}].\\


\textbf{Callback:}  \textit{Callback} o ``devoluci�n de llamada", es una funci�n en ROS que normalmente controla los mensajes, es decir, cada vez que llega un mensaje, ROS llama a su \textit{message manager} (administrador de mensajes) y le pasa el nuevo mensaje, para que este comience a trabajar con base en el mensaje recibido, en resumen ejecuta la funci�n al momento de recibir un mensaje.\\


\textbf{Filtrado de part�culas:} El \textit{filtrado de part�culas} consiste en estimar los estados internos en los sistemas din�micos cuando se hacen observaciones parciales, y las perturbaciones aleatorias que est�n presentes en los sensores y en el sistema din�mico. El objetivo del filtrado de part�culas es calcular las distribuciones posteriores de los estados de alg�n proceso de \textit{Markov}, dadas algunas observaciones ruidosas y parciales.

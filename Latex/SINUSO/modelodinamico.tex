
\section{Modelo Dinamico}

Miura y Shimoyama[23] estudiaron la aproximaci�n del modelo din�mico de un b�pedo a la de un p�ndulo invertido despu�s Kajita [24] estudi� la validaci�n de este m�todo a diferentes tipos de robots puesto que [23] trabaj� con un caso en particular.\\

Se aproxim� la dinamica del robot con el m�todo de p�ndulo invertido. La masa est� concentrada en el centro de masa (CoM) de el robot, y la base del p�ndulo coincide con el soporte del pie del robot, como se ilustra en la figura siguiente.


\begin{figure}[h]
	\centering
		\includegraphics{figs/dinamica.jpg}
	\caption{Pendulo Invertido}
	\label{fig:dinamica}
\end{figure}


Mediante las ecuaciones Lagrange:

\begin{equation}
	\tau_{x_a}(\theta)=m\left[g+\left(\frac{\tau_{x_a}(\theta)}{Lm}-gsen(\theta)\right)sen(\theta)-\frac{v^2}{L}cos(\theta)\right]x_a
\end{equation}
  
  
Cabe destacar que este modelo interactua solo con un torque m�ximo requerido dado por


  
  
\begin{equation}
	\tau_{x_a}\left(\theta\right)=\frac{mgx_{a}\left(1-sen^2\left(\theta\right)\right)}{1-\frac{sen\left(\theta\right)}{L}x_a}
\end{equation}

\begin{equation}
		\tau_{x_b}\left(\theta\right)=\frac{mgx_{b}\left(1-sen^2\left(\theta\right)\right)}{1-\frac{sen\left(\theta\right)}{L}x_b}
\end{equation}


Pero este modelo no considera las fuerzas laterales que son el resultado de conservar el equilibrio.


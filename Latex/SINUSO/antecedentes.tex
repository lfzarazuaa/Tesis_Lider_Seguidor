\chapter*{Antecedentes}


Las bases que existen en la actualidad para el desarrollo de humanoides caminantes fueron iniciadas por Vukobratovic en 1969, quien fue uno de los primeros en analizar la caminata de un b�pedo y establecer los criterios de balance.\\

Shuuji Kajita [15] dise�� y desarroll� un modelo casi ideal en 2D de un robot b�pedo. Kajita propuso por simplicidad que el Centro de Gravedad del robot se mueve horizontalmente y �l desarroll� una ley de control para el inicio, otra para la secuencia y otra para la finalizaci�n del proceso de caminado.\\

Zhen [1] propuso un esquema capaz de hacer que el robot escalara superficies inclinadas, con la ayuda de sensores de fuerza colocados en los pies del robot, con esto, las diferencias y rugosidades del piso pueden ser detectadas y entonces el robot puede compensar la inclinaci�n, con la ayuda de los movimientos adecuados de cada motor.\\

Shih [16] y Huang [17] usaron la interpolaci�n c�bica para generar las trayectorias tanto de cadera como de pie usando polinomios c�bicos para cualquier terreno que se presentara. El trabajo de Shih s�lo se bas� en problemas est�ticos, mientras que el de Huang propuso un m�todo para caminata din�micamente.\\

Kajita y Tani [18] usaron el m�dulo del p�ndulo invertido para lograr la caminata en un terreno robusto. Ellos dirigieron 2 experimentos: La fase del soporte simple de la pierna, y el cambio de la pierna de soporte; encontraron que para lograr un cambio suave del soporte de la pierna es necesario mantener una velocidad vertical como si se mantuviera por algunos instantes la fase de doble soporte. La Figura 1.1 muestra el robot que Kajita us� en sus experimentos.\\ 
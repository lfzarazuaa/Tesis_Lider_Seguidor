
\subsection{Mecanismo de la Marcha.}

Si se plantea el mecanismo por el que se produce la marcha, se debe tener en cuenta que el cuerpo humano al caminar se comporta como un sistema f�sico y como un organismo biol�gico, y por consiguiente est� sujeto a las leyes f�sicas del movimiento y a las leyes biol�gicas de la acci�n muscular. Se han estudiado los requerimientos energ�ticos de una persona adulta normal caminando a diferentes velocidades y se ha comprobado que toda persona tiene una velocidad de marcha de $4.5km/h$ la cual requiere un m�nimo de energ�a, para �sto, es interesante analizar los desplazamientos que sufre el centro de gravedad del cuerpo que aproximadamente se sit�a por delante de la segunda v�rtebra sacra.\\

Durante la marcha, el cuerpo sufre un r�tmico desplazamiento arriba-abajo, este desplazamiento vertical est� en �ntima relaci�n con la locomoci�n bipodal: en las fases de doble apoyo el centro de gravedad est� en el punto m�s bajo; en las fases de apoyo unilateral, el centro de gravedad alcanza su punto m�s alto, la distancia entre �stos dos puntos extremos es de 4 o 5 cm. Tambi�n se ha comprobado que el centro de gravedad, en su desplazamiento, describe una curva sinusoidal, que es la que demanda menor consumo energ�tico. Para conseguir �ste desplazamiento existe una serie de movimientos coordinados de la extremidad inferior. La pelvis, la cadera y la rodilla act�an coordinadamente para disminuir la amplitud de la curva, mientras que la rodilla, el tobillo y el pie trabajan para suavizar el cambio de sentido de la curva.

\begin{figure}[h]
	\centering
		\fbox{\includegraphics{figs/dos.jpg}}
	\caption{Movimiento del centro de gravedad.}
	\label{fig:dos}
\end{figure}
   
   
  La pelvis contribuye al desplazamiento del centro de gravedad con dos movientos: el primero en el plano horizontal, que son los movimientos vistos desde arriba o desde abajo del cuerpo humano [14] y el otro en el plano vertical, que son los movimientos vistos desde adelante o atr�s del cuerpo humano [14]. En el plano horizontal se realiza un movimiento de rotaci�n parecido al movimiento de un comp�s, que puede desplazarse sin cambiar la altura de la cruz. Si no se tuviera �ste movimiento, la caminata ser�a a modo del movimiento de unas tijeras, abriendo y cerrando las hojas. Al caminar se adoptan los dos tipos de movimiento, el del comp�s con la rotacion p�lvica y el de las tijeras con la flexoextenci�n de la cadera. Como lo muestran las figuras 1.13 y 1.14 respectivamente.\\
  
  
\begin{figure}
	\centering
		\fbox{\includegraphics[scale=1.3]{figs/tres.jpg}}
	\caption{Movimiento de la pelvis en el plano horizontal.}
	\label{fig:tres}
\end{figure}

  
  El movimiento en el otro plano consiste en la inclinaci�n de la pelvis hacia el lado de la pierna oscilante al igual que el movimineto de rotaci�n contribuye a disminuir el desplazamiento vertical del centro de gravedad. La rodilla tambi�n participa en la disminuci�n del desplazamiento del centro de gravedad al estar en discreta flexi�n en el momento en que el cuerpo pasa por encima de la pierna que apoya.\\
  
  
\begin{figure}[h]
	\centering
	
	\fbox{	\includegraphics[scale=1.2]{figs/cuatro.jpg}}
	\caption{Movimiento frontal de la pelvis.}
	\label{fig:cuatro}
\end{figure}

  
  Si estos tres movimientos no sucedieran, el desplazamiento vertical del centro de gravedad ser�a dos veces mayor; ahora bien, si s�lo participaran estos tres movimientos, la trayectoria descrita por el centro de gravedad ser�a de arcos interrumpidos. Si la caminata fuera con una rodilla r�gida, sin tobillo ni pie, el choque en el paso producir�a una desaceleraci�n brusca del centro de gravedad, en pocas palabras, el movimiento  de la rodilla, pie y tobillo sirven en el descenso como un sistema de amortiguamiento que coincide como lo plantea [4], y en el despegue act�a como un sistema de aceleraci�n del centro de gravedad.             
  